\documentclass{article}
\usepackage{amsmath}
\usepackage[margin=1.5in]{geometry}
\usepackage{hyperref}
\setlength{\parskip}{1em}

\begin{document}
Let me describe the system I have. For hardware I use, either 2 USRP
X300s(with 2 daughter boards each), or 2 pairs of N210 connected with
MIMO cable.

In each frame I transmit a PN1 sequence from TX1 while TX2 transmits
zeros. Once PN1 is transmitted I transmit $m$ zeros on both TX, where $m$ is
the delay of the pulse shaping filter in number of symbols.
Then I transmit PN2 from TX2 while TX1 transmits zeros. Once PN2 is
transmitted I transmit m zeros from both TX.
After this I transmit a "phasing" sequence consisting of alternating 1.0 
and -1.0. The length of this sequence is currently taken to be the
same as the length of the PN sequences.
After this I transmit the data - Alamouti encoded. Currently I send
another known sequence of $q$ symbols with BPSK modulation.
After data I transmit a sequence of m zeros from both the TX antennas.

Let the PN sequences be $p$ symbols long. Let the data be
$q$ symbols long. So in each frame, the transmissions from
each of the channels are

\begin{center}
\begin{tabular}{||c|c|c|c|c|c|c|c||}
  Symbol Count  &$p$    &$m$    &$p$    &$m$    &$p$      &$q$      &$m$    \\
  Channel 1     &PN1    &Zeros  &PN2    &Zeros  &Phasing  &Payload  &Zeros  \\
  Channel 2     &Zeros  &Zeros  &Zeros  &Zeros  &Phasing  &Payload  &Zeros  \\
\end{tabular}
\end{center}

\noindent
I use $p = 63, m = 3, q = 1024$. So I get a frame that is $L = 1222$
symbols long.

I construct the samples to be transmitted by passing the above frame
through a pulse shaping filter with interpolation k = 4, symbol delay m
= 3, and excess bandwidth beta = 0.5. I use 2 independent pulse shaping
filters on each of the TX streams. I call the execute functions together
on each of these shaping filters to ensure the parallelism in the sample
streams. The frame generator I use can be found at\cite{bib:framegen}.

After interpolation I get 2 streams of $k*L (=4888)$ samples each. In the
transmit thread, I ask for these streams of $k*L$ samples from the frame
generator, push it into the TX USRP configured to work in MIMO. In the
receive thread I collect 2 streams of $k*L$ samples from the RX USRP and
push it into frame synchronizer. These two things are repeated in a
loop. The front end which implements the transmit and the receive
threads can be found at\cite{bib:main}.

There are 2 samples streams fed into the synchronizer - one for each RX
antenna. The synchronizer has in total 4 detector modules. On each
stream I have one detector for PN1 and another detector PN2.

Each time the frame synchronizer is executed I pass 2 streams of $k*L$
samples. The frame may be starting at any arbit sample in this set of $k*L$ of samples.
So let us assume that PN1 detector on RX1 returns true at index $i_{11}$.
Let the index at which PN1 detector on RX2 returing true be $i_{12}$.
Let the PN2 detector on RX1 return true at index $i_{21}$.
Let the PN2 detector on RX2 return true at index $i_{22}$.

Since both RX1 and RX2 should be receiving same signals(except for a complex
scaling due to path differences), we also expect that $i_{12} \simeq i_{11}$
and $i_{21} \simeq i_{22}$. Since there is $p + m$ symbols between PN1 and PN2,
we expect $k*(p + m) = 4*(63 + 3) = 264$ sample difference between PN1 and PN2.
ie, we expect that, $i_{21} - i_{11} \simeq 264$, and $i_{22} - i_{21} \simeq 264$.
Since the PNs repeat every $k*L$ samples, and we are calling the execute
funtion on $k*L$ samples, we expect these indexes to not change by a lot over
the entire experiment. 

Well, I find that the indexes remain the same over the entire experiment. When
I run the experiment on N210s I find that $i_{12} = i_{11}$ and $i_{22} =
i_{21}$. Below is one set of values when I ran the experiment on the N210s.
\begin{itemize}
  \item $i_{11}$:  1268
  \item $i_{12}$:  1268
  \item $i_{21}$:  1538
  \item $i_{22}$:  1538
\end{itemize}

what I find amusing is that $i_{21} - i_{11} = 270$, where are we exepct it to
be 264. So I came to the conclusion that somehow a delay of 6 samples is
happening to TX2 chain. So I introduced a 6 sample delay in TX1 chain and then
things started to fall into place.

When I run experiment on X300s, there is some delay between the TX chanins as
in the previous case, and also a delay in the RX chains. When I introduced the
delay corrections in code there also things fell into place.

To explain how I am using the detector - modules. Currently I use the detector
module only to indicate the frame arrival and also to measure the coarse
frequency correction. I have 2 nco\_coarse, one for each channel. Each have
same frequency. I use 4 detector modules to come up with their own dphi\_hat
measurements. I average these 4 measurements and update the nco\_coarse
frequency. I don't use the tau\_hat measurements from the detector currently.
Instead what I have is a firpfb and its derivative with initial phase of
num\_filters/2 at the start. I use the average of the error values from the 2
channels to update the phase. I use power of the detected PN sequence(after pfb)
to compute the channel gains.

Some of my other colleagues are working on setting up MIMO with OFDM. But they
also have issues getting it to work, because of the delay issues.

\begin{thebibliography}{9}
\bibitem{bib:framegen}
  \url{https://github.com/manuts/liquid-gr/blob/tdma-ch1/src/alamouti/framegen.cc}

\bibitem{bib:main}
  \url{https://github.com/manuts/alamouti/blob/master/thread-main.cc}

\bibitem{bib:framesync}
  \url{https://github.com/manuts/liquid-gr/blob/tdma-ch1/src/alamouti/framesync.cc}

\end{thebibliography}
\end{document}
